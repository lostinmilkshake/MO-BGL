\documentclass[a4paper,12pt]{article}

\usepackage[english, russian]{babel}
\usepackage[utf8]{inputenc}
\usepackage{graphicx}
\usepackage{subfig}
\usepackage{amsmath}
\usepackage{listings}
\usepackage[top=2cm, bottom=1.5cm, left=2cm, right=1.5cm]{geometry}
\usepackage{mathtools}
\usepackage[numbered,framed]{matlab-prettifier}
\usepackage{filecontents}
\usepackage[T1]{fontenc}
\usepackage{listings,chngcntr}

\begin{document}
%\counterwithin{lstlisting}{}
\thispagestyle{empty} %чтобы не было номера на первой странице

\begin{centering}
	\textbf{
{\large МИНОБРНАУКИ РОССИИ\\
САНКТ-ПЕТЕРБУРГСКИЙ ГОСУДАРСТВЕННЫЙ\\
ЭЛЕКТРОТЕХНИЧЕСКИЙ УНИВЕРСИТЕТ\\
«ЛЭТИ» ИМ. В.И. УЛЬЯНОВА (ЛЕНИНА)\\
Кафедра САПР}\\
}
\end{centering}


\vspace{7cm}

\begin{centering}
\textbf{ {\large 
ЗАДАНИЕ\\
по лабораторной работе №2\\
по дисциплине «Методы оптимизации»\\
Тема: \guillemotleftМетоды одномерной оптимизации на основе поиска
стационарной точки\guillemotright\\
}}
\end{centering}

\vspace{4cm}

\begin{tabular}{l r}
    \textbf{ {\large Студенты:}}&\hspace{6cm} \textbf{ {\large Литвинов К.Л.}}\\
   \textbf{}&\hspace{6cm} \textbf{ {\large Гарцев Е.А.}}\\
   \textbf{{}}&\hspace{6cm} \textbf{\large{Бурков М.П.}}\\
    \textbf{ {\large Преподаватель:}}&\hspace{6cm} \textbf{ {\large Каримов А.И.}}\\
\end{tabular}

\vspace{6cm}


\begin{centering}
	{\large
Санкт-Петербург \\
2019 \\
}
\end{centering}
\end{document}
