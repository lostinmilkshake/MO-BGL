\documentclass[a4paper,12pt]{article}

\usepackage[english, russian]{babel}
\usepackage[utf8]{inputenc}
\usepackage{graphicx}
\usepackage{subfig}
\usepackage{amsmath}
\usepackage{listings}
\usepackage[top=2cm, bottom=1.5cm, left=2cm, right=1.5cm]{geometry}
\usepackage{mathtools}
\usepackage[numbered,framed]{matlab-prettifier}
\usepackage{filecontents}
\usepackage[T1]{fontenc}
\usepackage{listings,chngcntr}

\begin{document}
%\counterwithin{lstlisting}{}
\thispagestyle{empty} %чтобы не было номера на первой странице

\begin{centering}
	\textbf{
{\large МИНОБРНАУКИ РОССИИ\\
САНКТ-ПЕТЕРБУРГСКИЙ ГОСУДАРСТВЕННЫЙ\\
ЭЛЕКТРОТЕХНИЧЕСКИЙ УНИВЕРСИТЕТ\\
«ЛЭТИ» ИМ. В.И. УЛЬЯНОВА (ЛЕНИНА)\\
Кафедра САПР}\\
}
\end{centering}


\vspace{7cm}

\begin{centering}
\textbf{ {\large 
ЗАДАНИЕ\\
по лабораторной работе №2\\
по дисциплине «Методы оптимизации»\\
Тема: \guillemotleftМетоды одномерной оптимизации на основе поиска
стационарной точки\guillemotright\\
}}
\end{centering}

\vspace{4cm}

\begin{tabular}{l r}
\textbf{ {\large Преподаватель:}}&\hspace{6cm} \textbf{ {\large Каримов А.И.}}\\
\end{tabular}

\vspace{7cm}


\begin{centering}
	{\large
Санкт-Петербург \\
2019 \\
}
\end{centering}

\newpage
 
\subsection*{Цель работы}

Изучение среды MATLAB, создание программы для реализации одного из методов одномерного поиска на основе поиска стационарной точки: 

\begin{itemize}
	\item Метод Ньютона;
	\item Метод секущих;
	\item Метод Мюллера;
	\item Обратная параболическая интерполяция;
	\item Метод Брента-Деккера.
\end{itemize}


\subsection*{Основные теоретические положения}

Критические и стационарные точки функции определяются следующим образом.

\textbf{Критические точки} функции $f(x)$ -- точки, в которых производная $f'(x)$ не существует или обращается в нуль.

\textbf{Стационарные точки} функции $f(x)$ -- точки, в которых производная $f'(x)$ обращается в нуль.

При этом стационарные точки подрязделяются на:

\begin{itemize}
	\item экстремумы -- точки минимума или максимума;
	\item седловые точки -- точки, в которых производная нулевая, но минимум или максимум не достигается.
\end{itemize}


\textbf{Лемма Ферма} утверждает: производная $f'(x)$ дифференцируемой функции в точке экстремума равна нулю.
В соответствии с этой леммой, возможно использования метода нахождения нуля производной в качестве метода оптимизации. Для этого осуществляются следующие шаги:

1) Поиск ${x_i^*}: f''(x^*) = 0$

2) Осуществляется проверка: $x_i^*$ -- экстремум, если  

\begin{equation}
f'''(x^*) \neq 0
\label{lf1}
\end{equation}

 или 
 
 \begin{equation}
  f''(x^* - \epsilon)f''(x^* + \epsilon) < 0, 
  \label{lf2}
 \end{equation}
где $\epsilon > 0$ -- малое число (взаимозаменяемые условия).

\subsubsection*{Метод Ньютона}
Метод Ньютона минимизации функции является обобщением известного метода Ньютона отыскания корня уравнения 

\begin{equation*}
f'(x^*) = 0.
\end{equation*}


В качестве приближения $x_{k + 1}$ к минимуму $x^*$ берется точка, в которой производная 
$f'(x) = 0$, т. е. 
$f'(x_k) + \frac{f''(x_k)}{\Delta x_k} = 0$.
Таким образом, 

\begin{equation}
x_{k + 1} = x_k-\frac{f'(x_k)}{f ''(x_k)}.
\label{eq0}
\end{equation}

Метод Ньютона формулируется следующим образом.

\textbf{Подготовительный этап}. Берется начальная точка $x_1$, устанавливается счетчик итераций $k=1$.

\textbf{Основной этап}

1) Находится следующее значение $x_{k+1}$ по формуле \eqref{eq0}, проверяется критерий окончания поиска
В качестве критерия окончания поиска используются следующие сотношения:
\begin{align*}
|x_{k + 1}-x_k| &< \epsilon
\\
|f'(x_{k + 1})-f'(x_k)| &< \delta
\end{align*}

2) Если поиск не окончен, устанавливается новое значение счетчика $k := k + 1$, иначе -- проверяются критерии \eqref{lf1} или \eqref{lf2}, и если они удовлетвореня, найденное значение $x_{k+1}$ является искомым экстремумом.

\subsubsection*{Метод секущих}

Метод секущих предлагает заменить вторую производную 
$f''(x_k)$ в ньютоновской формуле её линейной аппроксимацией $(f'(x_k) - f'(x_{k - 1}))/(x_{k} - x_{k - 1})$. Тем самым,

\begin{equation*}
x_{k + 1} = x_k - \frac{f'(x_k)(x_k - x_{k - 1})}{f'(x_k) - f '(x_{k – 1})}.
\end{equation*}

Легко видеть, что $x_{k + 1}$ -- точка пересечения с осью абсцисс секущей прямой, проходящей через точки $x_{k}$ и $x_{k - 1}$.

\textbf{Псевдокод}

\textbf{Цикл} 

$x_{k + 1} = b_{k} - f'(b_k)(b_k - a_k)/(f'(b_k) - f'(a_k))$; 

\textbf{Если}

\quad $|f'(x_{k + 1})| \leq \epsilon$, \textcolor{gray}{//КОП}

\textbf{то}

\quad остановиться

\textbf{иначе} \textcolor{gray}{// Уменьшить интервал поиска минимума}


\quad \textbf{Если}

\quad\quad $ f'(x_{k + 1}) > 0$, 

\quad \textbf{то}

\quad\quad$a_{k + 1} = a_k$, $b_{k + 1} = x_{k + 1},$ 

\quad\textbf{иначе}


\quad\quad$a_{k + 1} = x_{k + 1}$, $b_{k + 1} = b_k$;


$k = k + 1$;


\textbf{Пока не выполнен КОП}


\subsubsection*{Метод Мюллера}


Метод Мюллера основан на методе секущих. Основная идея состоит в том, чтобы использовать не две, а три опорные точки, и строить секущие параболы, а не прямые. В качестве следующего приближения берется точка пересечения параболы и оси $x$.

Найдем параболу через три точки:

\begin{equation}
y = f(x_k) + w(x - x_k)+f[x_k,x_{k-1},x_{k-2}](x - x_k)^2,
\label{eq1}
\end{equation}
где

\begin{equation}
w = f[x_k,x_{k-1}] + f[x_k,x_{k-2}] - f[x_{k-1},x_{k-2}].
\label{eq2}
\end{equation}

Тогда очередная точка ищется по формуле

\begin{equation}
x_{k+1} = x_{k} - \frac{2f(x_k)}{w \pm \sqrt{w^2 - 4f(x_k)f[x_k,x_{k-1},x_{k-2}]}}.
\label{eq3}
\end{equation}

Все остальное в методе Мюллера аналогично методу секущих. \textbf{Разделенные разности} первого и второго порядка в уравнениях \eqref{eq1}--\eqref{eq3} находятся по следующим формулам. Разделенная разность первого порядка:

\begin{equation*}
f[x_0,x_1] = \frac{f(x_1) - f(x_0)}{x_1 - x_0}
\end{equation*}

Разделенная разность второго порядка:

\begin{equation*}
f[x_0,x_1,x_2] = \frac{f[x_1,x_2] - f[x_0,x_1]}{x_2 - x_0}
\end{equation*}

\subsubsection*{Метод обратной параболической интерполяции}

Ищет стационарную точку по формуле

\begin{multline*}
x_{n+1} = \frac{f'_{n-1}f'_{n}}{(f'_{n_2} - f'_{n - 1})(f'_{n - 2} - f'_{n})}x_{n-2} \; + \\ + \; \frac{f'_{n-2}f'_{n}}{(f'_{n-1} - f'_{n - 2})(f'_{n-1} - f'{n})}x_{n-1} \;+\; \frac{f'_{n-2}f'_{n-1}}{(f'_{n} - f'{n-2})(f'_n - f'_{n-1})}x_n
\end{multline*}

В остальном -- метод аналогичен методу Ньютона

\subsubsection*{Метод Брента-Деккера}

Метод Брента-Деккера объединяет метод средней точки, секущих и обратной квадратичной интерполяции для поиска нуля функции. Его реализация на псевдокоде представлена ниже.


\noindent 
\subsubsection*{Псевдокод}


\textbf{input} $a, b$, and (a pointer to) a function for $f$

calculate $f(a)$

calculate $f(b)$


\textbf{if} $f(a)f(b) \geq 0$ 
\textbf{then}

\quad exit function because the root is not bracketed.


\textbf{end if}


\textbf{if} $|f(a)| < |f(b)|$ 

\textbf{then}

\quad swap $(a,b)$

\textbf{end if}

$c := a$

set mflag

\textbf{repeat until} $f(b or s) = 0$ or $|b - a| < \epsilon$

\quad 
\textbf{if} $f(a) \neq f(c)$ and $f(b) \neq f(c)$ 
\textbf{then}

\quad \quad(inverse quadratic interpolation)

\quad\quad$ s:={\frac {af(b)f(c)}{(f(a)-f(b))(f(a)-f(c))}}+{\frac {bf(a)f(c)}{(f(b)-f(a))(f(b)-f(c))}}+{\frac {cf(a)f(b)}{(f(c)-f(a))(f(c)-f(b))}}$ 

\quad 
\textbf{else}

\quad \quad(secant method)

\quad \quad$s:=b-f(b){\frac {b-a}{f(b)-f(a)}}$ 


\quad 
\textbf{end if}

\quad 
\textbf{if} (condition 1) s is not between $ (3a+b)/4$ and $b$ or

\quad(condition 2) (mflag is set and $|s-b| \geq |b-c|/2$) or

\quad(condition 3) (mflag is cleared and  $|s-b| \geq |c-d|/2$) or

\quad(condition 4) (mflag is set and  $|b-c| < \delta$) or

\quad(condition 5) (mflag is cleared and  $|c-d| < \delta$) 
\textbf{then}

\quad \quad (bisection method)

\quad\quad$s:=\frac {a+b}{2}$

\quad\quad set mflag

\quad 
\textbf{else}

\quad\quad clear mflag

\quad 
\textbf{end if}

\quad calculate $f(s)$

\quad$d := c$ 

\quad$c := b$

\quad 
\textbf{if} $f(a)f(s) < 0$ 
\textbf{then}

\quad\quad $b := s $

\quad 
\textbf{else}

\quad\quad $a := s $

\quad 
\textbf{end if}

\quad 
\textbf{if} $|f(a)| < |f(b)|$ 
\textbf{then}

\quad\quad swap $(a,b) $

\quad end if

end repeat

output $b$ or $s$ (return the root)

\subsection*{Реализация в MATLAB}

При реализации методов следует придерживаться принципа: один метод, одна функция -- один \lstinline[style=Matlab-editor]{*.m}-файл.
Следует тщательно комментировать код и писать раздел описания файла в начале, чтобы можно было воспользоваться командой \lstinline[style=Matlab-editor]{help}. Для реализации методов понадобятся функции среды, аналогичные функциям, применяемым в лабораторной работе 1. Также потребуется функция экспорта растрового изображения
\lstinline[style=Matlab-editor]{export_fig}, которая устанавливается одноименным add-on'ом в соответствующей вкладке среды (для этого необходимо войти в аккаунт на официальном сайте Mathworks).

Пример правильного вызова команды \lstinline[style=Matlab-editor]{export_fig} следующий:

 \lstinline[style=Matlab-editor]{export_fig(gcf,'1.jpg','-transparent','-r300')}
 
 Здесь \lstinline[style=Matlab-editor]{gcf} отвечает за выбор текущего окна графика, \lstinline[style=Matlab-editor]{'1.jpg'} -- название файла, \lstinline[style=Matlab-editor]{'-transparent'} -- выбор прозрачного фона (белого для форматов типа *.jpg, не поддерживающих прозрачных цветов), \lstinline[style=Matlab-editor]{'-r300'} -- разрешение в 300 dpi, соответствующее заданному качеству печати.
 
 Также понадобится команда 
 
  \lstinline[style=Matlab-editor]{num2str(x)}
  
  Эта команда преобразует переменную  \lstinline[style=Matlab-editor]{x} в текст -- массив символов типа \lstinline[style=Matlab-editor]{char}.

Помимо реализации самих методов, в данной работе потребуется реализовать также и визуализацию процесса работы оптимизации. Пример соответствующего когда показан в листинге 1.

\newpage
\begin{lstlisting}[style=Matlab-editor, caption=Метод золотого сечения с визуализацией] 
function [xmin, fmin, neval] = goldensectionsearch2slides(f,interval,tol)
% goldensectionsearchvisual searches minimum using golden section
% [xmin, fmin, neval] = goldensectionsearch(f,interval,tol)
% f - an objective function handle
% interval = [a, b] - search interval
% tol - set for bot range and function value
    a = interval(1);
    b = interval(2);
    deltainterval = realmax; %value greater then tol
    L = b - a;
    neval = 0;
    Fi = (1 + sqrt(5))/2;
 
    % VISUALIZATION
    ctr = 1;
    deltaX = (b-a)/100;
    figure(3); hold on
    [miny, maxy] = drawplot(f,a,b,a,b);
    deltaY = abs(maxy - miny)/100;
    placelabel(a,0,deltaX,deltaY,ctr);
    placelabel(b,0,deltaX,deltaY,ctr);
    export_fig(gcf,'1.jpg','-transparent','-r300');
    
    % MAIN LOOP
    while L > tol
        L = b - a;
        
        x1 = b - L/Fi; 
        x2 = a + L/Fi;
        
        y1 = feval(f,x1);
        y2 = feval(f,x2);
        ctr = ctr + 1;
        
        if ctr < 10
            drawplot(f,a,b,x1,x2);
        end
        
        if y1 > y2
            a = x1;
            placelabel(x1,0,deltaX,deltaY,ctr);
            xmin = x2;
            fmin = y2;
        else
            b = x2;
            placelabel(x2,0,deltaX,deltaY,ctr);
            xmin = x1;
            fmin = y1;
        end
        
        neval = neval + 2;
        
        if ctr <= 10
           export_fig(gcf,[num2str(ctr),'.jpg'],'-transparent','-r300');
        end
    end
end
\end{lstlisting}

В Листинге 2 показана реализация используемой в коде функции \lstinline[style=Matlab-editor]{drawplot}.

\begin{lstlisting}[style=Matlab-editor, caption=Функция рисования графика] 
function [miny maxy] = drawplot(f,a,b,x1,x2)
    figure(3); 
    h = (b-a)/100;
    x = a:h:b;
    y = feval(f,x);
    
    miny = min(y);
    maxy = max(y);
    
    colp = hsv2rgb([rand(), 1, 0.5+0.5*rand()]);
    plot(x,y,'LineWidth',1,'Color',colp);
    scatter([a b],[feval(f,a), feval(f,b)],'Marker','o','MarkerFaceColor',colp,'MarkerEdgeColor',colp);
    xlabel('\itx');
    ylabel('\ity');
    line([a b],[0 0],'Color','k','LineWidth',1); %axis x
    col = hsv2rgb([rand(), 1, 0.5+0.5*rand()]);
    y1 = feval(f,x1);
    line([x1 x1],[0 y1],'Marker','s','Color',col,'LineWidth',1,'MarkerSize',4); 
    y2 = feval(f,x2);
    line([x2 x2],[0 y2],'Marker','s','Color',col,'LineWidth',1,'MarkerSize',4); 
end
\end{lstlisting}

В Листинге 3 показана реализация дополнительной функции \lstinline[style=Matlab-editor]{placelabel}.

\begin{lstlisting}[style=Matlab-editor, caption=Функция установки меток с нумерацией итераций] 
function placelabel(x,y,deltaX,deltaY,iternumber)
if iternumber <=10
    text(x - deltaX/2,y + 4*deltaY,num2str(iternumber));
end
end
\end{lstlisting}

Проверьте, как работает команда \lstinline[style=Matlab-editor]{help}, введя в командную строку

\lstinline[style=Matlab-editor]{>>help goldensectionsearch2slides}

Программа нарисует серию из 10 кадров, озаглавленных от ``1.jpg'' до ``10.jpg''. Пример кадра приведен на рисунке \ref{fig1}.

\begin{figure}
	\centering
	\includegraphics[width=12.3cm]{8.jpg}
	\caption{Кадр номер 8, полученный кодом \texttt{goldensectionsearch2slides}}
	\label{fig1}
\end{figure}

\subsection*{Содержание работы и отчета}

Лабораторная работа должна включать в себя *.m-файлы кодов на MATLAB для решения задачи и от 5 до 10 файлов изображений со стадиями выполнения каждого из заданных в варианте методов в формате *.jpg (количество файлов зависит от того, насколько читаемым является последнее изображение). Тип и формат визуализации может выбираться по усмотрению студента в зависимости от метода оптимизации. 

Отчет должен содержать титульный лист и разделы:

\begin{itemize}
\item Цель работы

\item Основные теоретические положения. 

\item Коды для решения задачи оптимизации.

\item Реализация отрисовки изображений для двух заданных методов и заданной целевой функции (в тексте приводится к качестве примера по одному изображению для каждого метода).

\item Выводы.
\end{itemize}

В приложении к данному документу находятся *.m-файлы, реализующие функции:
\begin{lstlisting}[style=Matlab-editor] 
f %objective function (2)
goldensectionsearch2slides %golden section search with visualization implementation
test_goldensectionsearch2slides %test file for goldensectionsearch2slides
\end{lstlisting}


В Таблице \ref{tab:1} приведены тестовые функции. Обратите внимание, что функция 1 неунимодальна!
\begin{table}
  \centering
    \begin{tabular}{| l | l | l |}
    \hline
     Номер & Функция & Точка $x^*$ \\ \hline
     1 & $y = x^2  -  10 \cos(0.5 \pi x) - 110$ & 0\\ \hline
     2 & $y = (x - 3)^2 - 50$  & 3 \\ \hline
     \end{tabular}
    \caption{Тестовые функции}
    \label{tab:1}
\end{table}

Таблица 2 содержит варианты заданий по вариантам. Номерами методов обозначены: 1 -- Больцано, 2 -- трехточечного деления, 3 -- золотого сечения, 4 -- Фибоначчи, 5 -- Ньютона, 6 -- секущих, 7 -- Мюллера, 8 -- обратная параболическая интерполяция, 9 -- Брента-Деккера.

\begin{table}
  \centering
    \begin{tabular}{| l | l | l |}
    \hline
     Вариант & Номера функций & Номера методов \\ \hline
      1 & 2& 1,5\\ \hline
	2 & 2 & 2,6\\ \hline
	3 & 2 & 4,7\\ \hline
	4 & 2 & 8,9\\ \hline
	5 & 1& 1,5\\ \hline
	6 & 1 & 2,6\\ \hline
	7 & 1 & 4,7\\ \hline
	8 & 1 & 8,9\\ \hline
	9 & 1 & 1,3\\ \hline
	10 & 2 & 1,3\\ \hline
	11 & 2 & 4,9\\ \hline
	12 & 2 & 5,8\\ \hline
	13 & 1 & 5,7\\ \hline
	14 & 1 & 6,7\\ \hline
	15 & 1 & 2,8\\ \hline
	16 & 1 & 3,9\\ \hline
     \end{tabular}
    \caption{Варианты}
    \label{tab:2}
\end{table}
\end{document}